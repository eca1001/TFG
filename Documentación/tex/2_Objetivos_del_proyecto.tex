\capitulo{2}{Objetivos del proyecto}

Este apartado explica de forma precisa y concisa cuales son los objetivos que se persiguen con la realización del proyecto. Se puede distinguir entre los objetivos generales, los objetivos de carácter técnico que plantea a la hora de llevar a la práctica el proyecto  y los objetivos de carácter personal.

\section{Objetivos generales}
\begin{itemize}
    \item Implementación de un sistema de recomendación basado en la encuesta de Stack Overflow.
    \item Emisión de recomendaciones de: bases de datos, entornos de desarrollo, herramientas, lenguajes, marcos de trabajo y plataformas.
    \item Creación de una aplicación web que pueda ser utilizada libremente por cualquier usuario sin necesidad de descargarse nada.
\end{itemize}

\section{Objetivos de carácter técnico}
\begin{itemize}
    \item \underline{Visualización}: diseñar una aplicación visualmente atractiva y fácil de comprender.  
    \item \underline{Online}: ejecutar la aplicación en un servidor web para que pueda ser utilizada por cualquier usuario sin necesidad de descargar nada.
    \item \underline{Interacción}: permitir a los usuarios interaccionar con la aplicación para así mejorar la experiencia de éste.
    \item \underline{Renovación}: la aplicación permite fácilmente actualizarse anualmente para tener los datos más recientes.
    \item \underline{Disponibilidad}: permite acceder a la aplicación en cualquier momento siempre que se tenga una conexión a internet.
    \item \underline{Usabilidad}: la aplicación tiene una interfaz muy clara y sencilla para permitir a cualquier usuario utilizarla y entenderla sin ningún problema.
    \item \underline{Asequibilidad}: es gratuita, por lo que no será necesario pagar para poder utilizarla.
    \item \underline{SCRUM}: metodología que controla la planificación del proyecto.
\end{itemize}

\section{Objetivos de carácter personal}
\begin{itemize}
    \item Aprender a desplegar aplicaciones en Heroku ya que hasta este momento no había utilizado esta herramienta.
    \item Aumentar mis conocimientos sobre creación de aplicaciones web creadas a partir de HTML y CSS.
    \item Aprender sobre la Teoría de Redes.
\end{itemize}