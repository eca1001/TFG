\apendice{Especificación de diseño}

\section{Introducción}
En este apéndice se van a detallar todos los aspectos relacionados con el diseño del proyecto.

\section{Diseño de datos}
Se ha utilizado la base de datos que Satck Overflow ofrece como resultado de la encuesta. La base de datos consta de 83440 filas y 48 columnas. Cada fila representa cada uno de las personas entrevistadas para la encuesta, y cada columna cada una de las preguntas.

De todas estas columnas se han elegido las siguientes:
\begin{itemize}
    \item Lenguajes con los que el usuario ha trabajado el último año.
    \item Bases de datos con las que el usuario ha trabajado el último año.
    \item Plataformas en la nube con las que el usuario ha trabajado el último año.
    \item Marcos de trabajo web con los que el usuario ha trabajado el último año.
    \item Herramientas con las que el usuario ha trabajado el último año.
    \item Entornos de desarrollo con los que el usuario ha trabajado el último año.
    \item Otros marcos de trabajo con los que el usuario ha trabajado el último año.
\end{itemize}


\section{Diseño procedimental}
En este apartado se va a describir el funcionamiento interno del proyecto mediante un diagrma de secuencias. \ref{fig:DiagramaSecuencias}.

\imagen{DiagramaSecuencias}{Diagrama de secuencias del funcionamiento interno}


\section{Diseño arquitectónico}
En este apartado se va describir los patrones que se han utilizado para el diseño del proyecto. \cite{wiki:patrones}

\subsection{Patrones estructurales}

\subsubsection{Fachada}
Este patrón pretende ofrecer al cliente una interfaz sencilla y disminuir el acoplamiento entre el subsistema y los clientes.

Se utiliza a la hora de presentar la aplicación y la comunicación de las páginas entre sí, ya que ofrece una interfaz sencilla que permite al usuario cambiar de pestañas y mantener el mismo umbral en todas ellas.

Su estructura es la siguiente: \ref{fig:Fachada}

\imagen{Fachada}{Patrón fachada}

\subsection{Patrones de comportamiento}
\subsubsection{Estado}
Este patrón permite a un objeto modificar su comportamiento cuando lo haga su estado interno.

En el proyecto se utiliza para crear nuevas redes en tiempo de ejecución una vez el usuario cambie el valor del umbral.

Su estructura es la siguiente: \ref{fig:Estado}

\imagen{Estado}{Patrón Estado}

\subsubsection{Observador}
Este patrón define una dependencia uno a varios para así en cuanto se modifique un estado todos sus dependientes sean notificados y actualizados automáticamente.

En el proyecto esto se utiliza para modificar las tablas, gráficos y redes que aparecen en cada una de las páginas. Una vez se modifica el umbral, se crea una nueva red y se notifica y actualiza en ese momento los objetos que aparecen en cada una de las páginas de las tecnologías. 

Su estructura es la siguiente: \ref{fig:Observador}

\imagen{Observador}{Patrón observador}

\subsubsection{Método plantilla}
Este patrón define el esqueleto de un algoritmo permitiendo a las subclases redefinir ciertos pasos del algoritmo sin cambiar su estructura general.

En el proyecto se utiliza para crear las páginas de las tecnologías. Las páginas heredan de una plantilla cambiando solamente en cada una de las tecnologías los datos a utilizar.

Su estructura es la siguiente: \ref{fig:Método plantilla}

\imagen{Método plantilla}{Patrón método plantilla}

\subsection{Patrones creacionales}
\subsubsection{Singleton}
Este patrón garantiza que una clase tenga tan solo una instancia, y que el acceso a la misma sea global. 

De esta forma al cliente tan solo le llega sólo un objeto (página web en este caso), y es este el que se modifica en cuanto se produce un cambio, no creando nuevas instancias.

Su estructura es la siguiente: \ref{fig:Singleton}

\imagen{Singleton}{Patrón singleton}